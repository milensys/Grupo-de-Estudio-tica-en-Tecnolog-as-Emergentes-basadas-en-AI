\chapter{Ejemplos de Código}
%\label{chapter:codigo-fuente}

\emph{En este capítulo se presentan ejemplos de anexos de código en Python que ilustran conceptos clave relacionados con la ética en tecnologías emergentes. El primero aborda simulaciones atmosféricas, mientras que el segundo se centra en la evaluación de sesgos en modelos de AI. Ambos son representativos de aplicaciones prácticas con implicaciones éticas.}

\section{Simulación Atmosférica para Tecnologías Emergentes}

\emph{El siguiente ejemplo muestra cómo calcular propiedades atmosféricas según la altitud. Este código es relevante para drones y sistemas autónomos que dependen de simulaciones atmosféricas.}

\begin{lstlisting}[language=Python, caption={Cálculo de propiedades atmosféricas basado en ISA}, label={lst:atmosfera}]
"""
ISA Calculator: Calcula temperatura, densidad, presión y velocidad del sonido
en función de la altitud.
"""

import math
g0 = 9.80665
R = 287.0
layer1 = [0, 288.15, 101325.0]
alt = [0, 11000, 20000, 32000, 47000, 51000, 71000, 86000]
a = [-.0065, 0, .0010, .0028, 0, -.0028, -.0020]

def atmosphere(h):
    for i in range(0, len(alt)-1):
        if h >= alt[i]:
            layer0 = layer1[:]
            layer1[0] = min(h, alt[i+1])
            if a[i] != 0:
                layer1[1] = layer0[1] + a[i]*(layer1[0]-layer0[0])
                layer1[2] = layer0[2] * (layer1[1]/layer0[1])**(-g0/(a[i]*R))
            else:
                layer1[2] = layer0[2] * math.exp((-g0/(R*layer1[1]))*(layer1[0]-layer0[0]))
    return [layer1[1], layer1[2]/(R*layer1[1]), layer1[2], math.sqrt(1.4*R*layer1[1])]
\end{lstlisting}

\textbf{Aplicación ética:} Este código puede ser utilizado para garantizar la seguridad y sostenibilidad de sistemas autónomos como drones, al prever condiciones atmosféricas extremas que podrían impactar su operación.

\section{Evaluación de Sesgos en Modelos de IA}

\emph{El siguiente ejemplo aborda la evaluación de sesgos en modelos predictivos de inteligencia artificial. Esto es crucial para garantizar la equidad en decisiones automatizadas.}

\begin{lstlisting}[language=Python, caption={Evaluación de sesgos en modelos predictivos.}, label={lst:sesgo}]
"""
Evaluación de Sesgos en Modelos Predictivos:
Este código evalúa la equidad de un modelo de clasificación usando paridad demográfica.
"""

import numpy as np

def demographic_parity(y_true, y_pred, sensitive_attribute):
    """
    Calcula la paridad demográfica para un modelo de clasificación.
    """
    groups = np.unique(sensitive_attribute)
    parity = {}
    for group in groups:
        mask = sensitive_attribute == group
        parity[group] = np.mean(y_pred[mask])
    return parity

# Ejemplo de uso
y_true = np.array([1, 0, 1, 0, 1, 0])
y_pred = np.array([1, 0, 1, 0, 0, 0])
sensitive_attribute = np.array(["A", "A", "B", "B", "A", "B"])

parity = demographic_parity(y_true, y_pred, sensitive_attribute)
for group, value in parity.items():
    print(f"Grupo {group}: Paridad demográfica = {value:.2f}")
\end{lstlisting}

\textbf{Aplicación ética:} Este ejemplo demuestra cómo detectar posibles sesgos en decisiones automatizadas, un tema central en justicia algorítmica.

---

