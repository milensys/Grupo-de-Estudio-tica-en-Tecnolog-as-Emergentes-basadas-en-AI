\chapter{División de Tareas}
%\label{chapter:title}

\emph{Si se requiere una división de tareas, a continuación se presenta una plantilla adaptada para un grupo de estudio en ética en tecnologías emergentes basadas en AI y Python. Esta tabla puede ser personalizada según las necesidades del grupo.}
\begin{table}[htb]
    \setlength\extrarowheight{4pt}
    \centering
    \caption{Distribución de las tareas}
    \label{tab:taskdivision}
    \begin{tabularx}{\textwidth}{lXX}
        \toprule
        & Tarea & Miembro(s) \\
        \midrule
        & Resumen general & Urania ... \\
        
        Capítulo 1 & Introducción: Panorama de la ética y tecnologías emergentes & Urania \\
        Capítulo 2 & Impacto de Tecnologías Emergentes: Transformaciones sociales & miembro2 \\
        Capítulo 3 & Explicabilidad e Interpretabilidad de la IA & miembro3 \\
        Capítulo 4 & Justicia Algorítmica y Sesgos en la IA & miembro4 \\
        Capítulo 5 & IA Generativa y Sistemas Autónomos: Implicaciones éticas & miembro1, miembro2 \\
        Capítulo 6 & Soberanía Digital y Gobernanza Global & miembro3 \\
        Capítulo 7 & Impacto Ambiental y Sostenibilidad Tecnológica & miembro4 \\
        Capítulo 8 & Identidad Digital, Ética Biométrica y Derechos Humanos & miembro1 \\
        Capítulo 9 & Anticipación y Mitigación de Riesgos Globales & miembro2 \\
        Capítulo 10 & Ética en Biotecnología y Nanotecnología & miembro3, miembro4 \\
        Capítulo 11 & Economía de Datos y Ética del Consumo Tecnológico & miembro1 \\
        Capítulo 12 & Biopiratería y Secuencias Genéticas Digitales & miembro2 \\
        Capítulo 13 & Ética del IoT e Inteligencia Ambiental & miembro3 \\
        Capítulo 14 & Computación Neuromórfica y Ética de la Imitación Biológica & miembro4 \\
        Capítulo 15 & Interfaces Cerebro-Computadora y Transhumanismo & miembro1 \\
        Capítulo 16 & Soberanía Energética y Computación Verde & miembro2 \\
        Capítulo 17 & Ética en la Automatización Masiva & miembro3 \\
        Capítulo 18 & Ética del Metaverso & miembro4 \\
        Capítulo 19 & Reflexión Filosófica: ¿Deberíamos hacerlo porque podemos? & miembro1 \\
        Capítulo 20 & Conclusiones y Visión a Largo Plazo & miembro2, miembro3 \\

        \midrule
        & Investigación & miembro1, miembro4 \\
        & Diseño de simulaciones en Python & miembro2 \\
        & Desarrollo de gráficos y figuras explicativas & miembro3 \\
        & Diseño y Formato del Documento & miembro1, miembro4 \\
        & Revisión de contenido técnico y ético & miembro2, miembro3 \\
        & Supervisión general y edición final & miembro1 \\
        \bottomrule
    \end{tabularx}
\end{table}
