\begin{refsection}[references/chapter-2.bib]
\chapter{Impacto de las Tecnologías Emergentes}
\label{chapter:chapter-2}

\section{Introducción}
Las tecnologías emergentes, como la inteligencia artificial, la automatización y la digitalización, están reconfigurando de manera profunda cómo trabajamos y vivimos. ¿Qué significa esta transformación para los conceptos tradicionales de empleo? Este capítulo explora los impactos sociales y económicos de estas tecnologías, cuestionando si estamos preparados para un futuro donde el trabajo y su significado están en constante cambio.

\section{Transformación del Trabajo y Disrupción Laboral}
¿Cómo están reconfigurando estas tecnologías las dinámicas laborales? Desde la automatización hasta los cambios en las modalidades de empleo, los efectos son amplios y diversos:
\begin{itemize}
    \item \textbf{Automatización y Robotización:} La integración de robots colaborativos (\textit{cobots}) en manufactura y logística plantea interrogantes sobre el lugar del ser humano en estos procesos \cite{WOS:000668201400001}.
    \item \textbf{Desigualdad Laboral:} ¿Están las tecnologías reforzando o mitigando las desigualdades preexistentes? Los estudios muestran efectos desiguales según género y raza \cite{WOS:001108050800001}.
    \item \textbf{Nuevas Modalidades de Trabajo:} Con el auge del trabajo remoto, ¿cómo se redefine el bienestar y la productividad en entornos virtuales? \cite{WOS:001195543500001}.
\end{itemize}

\section{El Concepto de Trabajo Decente en la Era Digital}
¿Es posible mantener el ideal del trabajo decente en un mundo digitalizado? Este concepto, promovido por la OIT, enfrenta retos importantes:
\begin{itemize}
    \item \textbf{Equidad Salarial y Condiciones Laborales:} La creciente automatización plantea dilemas sobre cómo garantizar un ingreso digno \cite{WOS:000691461300001}.
    \item \textbf{Reskilling y Upskilling:} ¿Cómo preparar a la fuerza laboral para los trabajos del futuro? La capacitación es clave para la sostenibilidad del empleo \cite{WOS:001235231900001}.
    \item \textbf{Bienestar Laboral:} ¿Qué estrategias son necesarias para proteger la salud física y mental en entornos altamente automatizados? \cite{WOS:000569348200001}.
\end{itemize}

\section{Estudios de Caso por Sector}
Sectores clave ejemplifican el impacto de estas tecnologías:

\subsection{Salud}

Un estudio reciente en India mostró que un sistema de IA detectó enfermedades cardíacas con una precisión del 95\%, pero el personal médico expresó preocupaciones sobre la pérdida de habilidades humanas en la interpretación de diagnósticos \cite{WOS:000701263900009}.


\subsection{Educación}
Con la incorporación de IoT y robótica en las aulas, ¿cómo cambiará la interacción entre docentes y estudiantes? \cite{WOS:001172649500001}.

\subsection{Manufactura Avanzada}

En un caso en Japón, la implementación de cobots en fábricas textiles redujo los costos de producción en un 40\%, pero también desplazó al 20\% de la fuerza laboral. Esto generó programas de reskilling enfocados en el diseño y mantenimiento de cobots \cite{WOS:000668201400001}.

\subsection{Servicios Financieros}
Los chatbots y sistemas de IA están transformando la atención al cliente y las decisiones de crédito. Sin embargo, el empleo en call centers y sucursales está disminuyendo rápidamente. Ejemplo: El impacto de algoritmos automatizados en decisiones de préstamos \cite{WOS:001035901600001}.

\subsection{Construcción}
La robótica y la impresión 3D han comenzado a reemplazar tareas manuales, lo que aumenta la productividad pero reduce la necesidad de trabajadores poco calificados. ¿Cómo pueden los programas de reskilling abordar esta brecha?

\subsection{Logística y Transporte}
El uso de vehículos autónomos y drones plantea preguntas sobre el futuro de conductores y trabajadores de almacenes. Ejemplo: Implementación de cobots en centros de distribución \cite{WOS:000668201400001}.



\section{Contexto Interdisciplinario}
Las tecnologías emergentes no operan en un vacío. Su impacto en el trabajo y la sociedad está moldeado por factores económicos, sociales y culturales. Por ejemplo:
\begin{itemize}
    \item \textbf{Dimensión Económica:} Los avances tecnológicos impulsan la productividad, pero también reconfiguran las estructuras del mercado laboral, generando desigualdades si no se gestionan adecuadamente.
    \item \textbf{Dimensión Social:} La integración de tecnología puede fomentar o socavar la cohesión social dependiendo de cómo se distribuyan los beneficios y costos.
    \item \textbf{Dimensión Cultural:} La aceptación de nuevas tecnologías depende en gran medida de cómo se alineen con valores y normas locales.
\end{itemize}


\section{Indicadores de Impacto}
Para analizar el impacto de las tecnologías emergentes en el trabajo decente, se pueden utilizar métricas como:
\begin{itemize}
    \item \textbf{Índice de Automatización Laboral:} Mide el porcentaje de tareas que han sido sustituidas por máquinas en diferentes industrias.
    \item \textbf{Índice de Reskilling:} Evalúa la proporción de trabajadores que han recibido capacitación para adaptarse a nuevos roles tecnológicos.
    \item \textbf{Índice de Bienestar Laboral:} Refleja niveles de satisfacción, salud mental y seguridad percibida por los empleados en entornos digitales.
\end{itemize}

\section{Impacto Global y Regional}
El efecto de las tecnologías emergentes en el trabajo depende del contexto regional:
\begin{itemize}
    \item \textbf{Países Desarrollados:} Alta adopción de automatización, desplazando trabajos repetitivos pero creando roles tecnológicos especializados.
    \item \textbf{Países en Desarrollo:} Riesgo de exclusión tecnológica para sectores menos digitalizados. Ejemplo: Efectos de la automatización en manufactura textil en India \cite{WOS:000642127500001}.
    \item \textbf{Disparidades Regionales:} En países como Sudáfrica, la automatización exacerba la desigualdad laboral en sectores con alta informalidad \cite{WOS:000701263900009}.
\end{itemize}



\section{Consideraciones Éticas}
El impacto de las tecnologías emergentes en el trabajo plantea dilemas éticos importantes:
\begin{itemize}
    \item \textbf{Responsabilidad Social:} ¿De quién es la responsabilidad de reentrenar a los trabajadores desplazados por la automatización?
    \item \textbf{Equidad:} ¿Cómo garantizamos que los beneficios de la innovación tecnológica lleguen a todos los sectores de la sociedad?
    \item \textbf{Consentimiento Informado:} ¿Los empleados tienen suficiente información para comprender cómo las tecnologías afectan su rol y privacidad?
\end{itemize}


\section{Actividad Opcional}

Leer artículos con modelos que analicen los efectos de la automatización en diferentes industrias, destacando los sectores más afectados y los trabajos emergentes.



\section{Conclusión del Capítulo 2}
Aunque las tecnologías emergentes están transformando el panorama laboral de manera irreversible, este cambio no tiene por qué ser una amenaza. Algunas estrategias clave incluyen:
\begin{itemize}
    \item \textbf{Políticas Públicas Inclusivas:} Promover marcos regulatorios que protejan a los trabajadores más vulnerables mientras fomentan la innovación.
    \item \textbf{Educación Continua:} Implementar programas de formación que combinen habilidades técnicas y sociales, preparando a la fuerza laboral para el futuro.
    \item \textbf{Alianzas Público-Privadas:} Fomentar la colaboración entre gobiernos, empresas y organizaciones sociales para distribuir equitativamente los beneficios de la tecnología.
\end{itemize}




% Include all entries from chapter-2.bib
\nocite{*}

% Print bibliography for Chapter 2
\printbibliography[heading=subbibliography, title={Bibliografía del Capítulo 2}]
\end{refsection}
