\begin{refsection}[references/chapter-6.bib]
\chapter{Soberanía Digital, Privacidad y Gobernanza de la IA}
\label{chapter:chapter-6}


%% tema del capitulo comentado, no se ve cuando se compila
\begin{comment}
\begin{enumerate}
\item \textbf{Soberanía Digital, Privacidad y Gobernanza de la IA (Capítulo 6):}  
\begin{itemize}
    \item Contexto de riesgos de la vigilancia masiva, soberanía digital y gobernanza descentralizada.
    \item Herramientas como blockchain para proteger datos y derechos digitales.
    \item Dimensión de explicabilidad: Visualización de datos para facilitar la gobernanza participativa.
    \item Actividad: Desarrollar un sistema de gestión de identidad descentralizada con \texttt{Flask} y blockchain.
\end{itemize}
\end{enumerate}
\end{comment}





\nocite{*}


\printbibliography[heading=subbibliography, title={Bibliografía del Capítulo 6}]
\end{refsection}
